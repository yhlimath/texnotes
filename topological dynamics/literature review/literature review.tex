\documentclass{article}
\usepackage{amsmath}
\usepackage{mathrsfs}
\usepackage{amssymb}
\usepackage{dcolumn}
\usepackage{graphicx}
\usepackage{fancybox}
\usepackage{amsmath}
\usepackage{amsthm}
\usepackage{tikz-cd}

\usepackage{CJK}

\newtheorem{definition}{Definition}
\newtheorem{theorem}{Theorem}
\newtheorem{proposition}{Proposition}
\newtheorem{lemma}{Lemma}
\newtheorem{corollary}{Corollary}
\newtheorem{example}{Example}
\newtheorem{remark}{Remark}


\renewcommand{\today}{\number\year.\number\month.\number\day}

\usepackage{CJK}




\begin{document}
\begin{CJK}{UTF8}{gbsn}

\title{Literature Review}
\author{Li Yueheng, Ma Yue, Yi Zichun}
\maketitle

\section{Literature Review}


A dynamical system could have various subsystems. An intriguing question is whether there exist some subsystems that may serve as approximation of the others. Here we introduce the concept of genericity.

The definition of genericity below was given by Hochman\cite{Hochman}.


\begin{definition}[Genericity]
	In a topological dynamical system, a property $\textbf{P}$ is generic if the set of subsystems satisfying $\textbf{P}$ contains a dense $G_\delta$ set in the space of all subsystems equipped with the Hausdorff metric.
\end{definition}

Dense subsets of the set of subsystems are also worth noticing, as every generic subset is dense. We also wish to find properties that define a dense set of subsystems.

Our goal is to find generic or dense properties and study the collection of subsystems defined by the property.

It was shown that topological entropy, a measure of complexity of dynamical systems, can be used to define generic properties. Hochman\cite{Hochman} proved that in shift systems, zero entropy shifts are generic. Further results were given by Frisch et.al\cite{Frisch} that entropy is upper semi-continuous, and that the shifts with entropy $c$ are generic in the set of shifts with entropy at least $c$.

\
\vspace{3ex}

The definition of genericity given above is quite straightforward and useful in the case of shift system, which is our main focus. But a general definition is also needed to understand other discussion about genericity. The next definition of genericity includes the former one as a special case:

\begin{definition}[Genericity]
	 Let $\mathscr{S}$ be the set of systems we discuss equipped with a topology. A property is generic if the subset of systems satisfying the property contains a $G_\delta$ subset.
\end{definition}

Based on the definition above, we may consider genericity in a more subtle manner. If the phase space $X$ is equipped with a probability measure $\mu$, it becomes a probability space $(X,\mathscr{F},\mu)$. We consider the measure preserving transforms on $X$. Each transform defines a dynamical system on $X$. While as they preserve measure, each transform induces an automorphism of the probability space. Interpreting them as positive linear functionals on  $\mathscr{F}$, we could introduce the weak-*topology on the space of measure preserving transforms, and therefore  the space of measure preserving dynamical systems. Now we may seek for generic properties in the space of measure preserving systems. 

\vspace{3ex}

We are concerned with the following questions:

1.Genericity in zero entropy systems: The set of systems with zero entropy is generic, but it still has complex structure. Is it possible to sort out some structures that preserve the generic property?

2.Relative genericity problem: Any system with finite entropy can be extended into a subsystem of the shift system. There might be multiple extensions for one system. We will look for generic properties in the space of all extensions.


\section{Basic Concepts}
In this section we elaborate some concepts and results mentioned in the previous section.
\subsection{Shift System}

Let $A:=\{0,1,\dots ,n-1\}$ be an \textit{alphabet} equipped with discrete topology. $X:=A^\mathbb{Z}$ is the set of infinite sequences of letters from the alphabet, denoted as $(a_i)$. A product topology is defined on it. Finally we define a shift mapping $\sigma$ on $X$:
\begin{align*}
	\sigma:X&\rightarrow X \\
	(a_i)&\mapsto(a_{i+1})
\end{align*}
Therefore $\sigma$ moves every element of a sequence leftward and is an open homeomorphism.
$(X,\sigma)$ is the shift system.

We'd like to introduce a metric $d$ on $X$, the metric topology induced by which coincides with the product topology defined above so $X$ now becomes a compact metric space.
\begin{align*}
	&\forall x,y\in X \\
	d(x,y)&:=\frac{1}{1+max\{j:x_i=y_i,\forall |i|\leq j\}}
\end{align*}

\begin{remark}
	Define finite sequences of elements of $A$ as \textbf{words}, the set of all words as \textbf{the language generated by $A$}.
\end{remark}


\begin{definition}[Irreducibility]
	A shift system $X$ is irreducible if for each pair of words $u,v$ presented in $X$, there exists a $X$ presented word $w$ such that the concatenation $uwv$ presents in $X$.
\end{definition}

For shift systems, irreducibility is equivalent to transitivity.


\begin{definition}[Shift]
	A shift is a closed, nonempty $\sigma$ invariant subsets of $X$. They are subsystems of $(X,f)$. 
\end{definition}

Let $\mathcal{S}:=\{S:S\mbox{ is a shift of }X\}$ the set of all shifts in $(X,f)$. We introduce the Hausdorff metric $d_H$ so $\mathcal{S}$ becomes a metric space $(\mathbb{S},d_H)$.



\begin{definition}[Hausdorff Metric]
	Let $X$ be a compact metric space, $Y_0,Y_1$ be nonempty, closed subsets of $X$. define the Hausdorff metric as
	\begin{align*}
		d_H(Y_0,Y_1):=\inf\{\varepsilon:\forall y\in Y_i, \exists y'\in Y_{1-i} \mbox{s.t. }d(y,y')<\varepsilon ,i=0,1\}
	\end{align*}
\end{definition}



\begin{proposition}
	In the topology induced by the Hausdorff metric, $\{Y_n\}\rightarrow Y$ iff $Y$ consists of the limit points of $\{y_n\}$, where $y_n\in Y_n$, and such convergence principle characterizes this topology.
\end{proposition}

It is worth noticing that  $(\mathcal{S},d_H)$ is a compact metric space. Our aim is to find generic properties in this space.

\subsection{Topological Entropy}
We consider a discrete dynamical system on a compact metric space$(X,f)$, its topological entropy describes the complexity of orbits in the system. To be specific, we consider the orbits corresponding to different initial values. If a small perturbation of the initial value results in significant change of its corresponding orbit, the system is considered complex since it would be hard in that case to find "typical" orbits that can approximate all others.


First we introduce a metric $d_n^f$ on $X$ that measures how differently the system evolves given different initial values in a given amount of time $n$.


\begin{align*}
    &d_n^f:X\times X\rightarrow R^+ \\
	&d_n^f(x,y):=\max_{0\leq i\leq n-1}d(f^i(x),f^i(y))
\end{align*}
This is indeed a metric and induces the same topology as the original metric on $X$ does.

\vspace{3ex}

Let $K$ be a compact subset of $X$, $S$ another subset and $\varepsilon$ a positive number. Define $S$ as a $(\varepsilon,n)$spanning set of $K$ if for each point $k\in K$, there exists $s\in S$ s.t. $d_n^f(k,s)<\varepsilon$.

Therefore, every orbit originated in $K$ can be approximated by an orbit originated in $S$ up to an error of $\varepsilon$ within time $n$.

The compactness of $K$ yields that there exists a finite $(\varepsilon,n)$ spanning set of $K$. Therefore we may consider the size of $S$. If there exists a $S$ with small ordinal, then the orbits in K are relatively simple since they can be approximated by a few typical orbits in $S$.

We define a number to measure the regularity of orbits in $K$ in the following way:

\begin{align*}
	minspan_n(\varepsilon,K,f):=\min\{|S|:S\subset X\mbox{ is a }(\varepsilon,n) \mbox{ spanning set of }  K\}
\end{align*}
	
This number describes how much at least typical orbits are needed to approximate all the orbits in $K$ given error rate and time.

\vspace{3ex}

On the other hand, a similar definition reviews the complexity of orbits in $K$.

We say $S$ is a $(\varepsilon,n)$ separated set of $K$ if any pair of orbits starting from distinct points of $S$ are distanced at least $\varepsilon$ within time $n$. Similarly we define a number

\begin{align*}
	maxsep_n(\varepsilon,K,f):=\max\{|S|:S\subset X \mbox{is a } (\varepsilon,n) \mbox{separated set of } K\} 
\end{align*}

We further define 

\begin{align*}
	r(\varepsilon,K,f)&:=\limsup_{n\rightarrow\infty}\frac{1}{n}\log minspan_n(\varepsilon,K,f) \\
	s(\varepsilon,K,f)&:=\limsup_{n\rightarrow\infty}\frac{1}{n}\log maxsep_n(\varepsilon,K,f) \\
	h(K,f)&:=\lim_{\varepsilon\rightarrow0}r(\varepsilon,K,f)=\lim_{\varepsilon\rightarrow0}s(\varepsilon,K,f)
\end{align*}

\begin{definition}[Topological Entropy]
	$h(K,f)$ is the topological entropy of $(K,f)$.
\end{definition}


\begin{definition}[Semi-Continuity]
	A function $f:E\rightarrow \mathbb{R}$ is upper semi-continuous if for all$x \in E, b>f(x)$, there exists  a neighborhood of x $U$ such that $f(U)\subset (-\infty,b)$.
	
	Lower semi-continuity can be defined similarly.
\end{definition}

It is shown by Frisch\cite{Frisch} et.al that the entropy function is upper semi-continuous.

\begin{theorem}
	The entropy map $h:\mathfrak{S}\rightarrow \mathbb{R}_{>0}$ is upper semi-continuous.
\end{theorem}

Based on this result, they further proved that

\begin{theorem}
	Shifts with entropy $c$ are dense in $\mathfrak{S}_{h\geq c}$, and are in fact generic.
\end{theorem}

\subsection{Measure Preserving Transforms and Genericity}

In this section we introduce a probability measure $\mu$ on $X$, which makes it a probability space $(X,\mathscr{F},\mu)$. We consider all the measure preserving transforms $T$ on $X$, i.e.
\begin{align*}
	\forall A\in\mathscr{F},\\
	\mu(A)=\mu(T^{-1}A)
\end{align*}


\begin{definition}[Automorphism of a Probability Space]
	A measure preserving transform $T$ on $(X,\mathscr{F},\mu)$ naturally induces an automorphism $\varphi$ of $\mathscr{F}$ by setting
	\begin{align*}
	\forall A\in\mathscr{F},\\
		\varphi(A):=T^{-1}A
	\end{align*}
\end{definition}

Now we understand $T$ as the automorphism $\varphi$, which, by the Riesz representation theorem, is a positive linear functional on $(X,\mathscr{F})$. Therefore we may define a weak-*topology on the set of measure preserving transforms.

A famous result was established by Halmos, to understand this result, we need to first introduce the weakly mixing property.

\begin{definition}[Weakly Mixing Transform]
	$(X,\mathscr{F},\mu)$ is a probability space. Let $T:X\rightarrow X$ be a measure preserving transform, $T$ is weakly mixing if $\forall A,B\in\mathscr{F}$,
	\begin{align*}
		\frac{1}{n}\sum_{k=0}^n|\mu(A\cap T^{-k}B)-\mu(A)\mu(B)|\rightarrow 0,n\rightarrow \infty
	\end{align*}
\end{definition}

The following famous theorem was established by Halmos\cite{Halmos} and was recently furthered by Glasner and Weiss\cite{Glasner}:

\begin{theorem}[Halmos]
	In the space of measure preserving transforms equipped with the weak-*topology, weakly mixing is a generic property.
\end{theorem}



\begin{theorem}[Glasner, Weiss]
	Relative weak mixing is generic
\end{theorem}





\begin{thebibliography}{99}
\bibitem{Hochman}Hochman, M. (2008). "Genericity in topological dynamics." Ergodic Theory and Dynamical Systems 28(1): 125-165.
\bibitem{Frisch}Frisch, J. and O. Tamuz (2016). "Symbolic dynamics on amenable groups: the entropy of generic shifts." Ergodic Theory and Dynamical Systems 37(4): 1187-1210.
\bibitem{Halmos}Halmos P R. Lectures on Ergodic Theory. Publications of the Mathematical Society of Japan, No. 3. Tokyo: Math
Soc Japan, 1956
\bibitem{Glasner}Glasner, E. and B. Weiss, Relative weak mixing is generic. Science China Mathematics, 2018. 62(1): p. 69-72.
\bibitem{Vries}Jan Vries - Topological Dynamical Systems. ISBN 978-3-11-034073-0
\bibitem{Mike}Schnurr, M., Generic properties of extensions. Ergodic Theory and Dynamical Systems, 2018. 39(11): p. 3144-3168.

\end{thebibliography}


\end{CJK}
\end{document}
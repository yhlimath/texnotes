\documentclass{article}
\usepackage{amsmath}
\usepackage{mathrsfs}
\usepackage{amssymb}
\usepackage{dcolumn}
\usepackage{graphicx}
\usepackage{fancybox}
\usepackage{amsmath}
\usepackage{amsthm}
\usepackage{amsrefs}
\usepackage{tikz-cd}

\usepackage{CJK}


\newtheorem{definition}{Definition}
\newtheorem{theorem}{Theorem}
\newtheorem{proposition}{Proposition}
\newtheorem{lemma}{Lemma}
\newtheorem{corollary}{Corollary}
\newtheorem{example}{Example}
\newtheorem{remark}{Remark}

\renewcommand{\today}{\number\year.\number\month.\number\day}






\begin{document}


\title{Symbolic Dynamics on $\mathbb{Z}$}
\author{Li Yueheng s2306706}

\maketitle


\section{Introduction}
\subsection{Notations}
In this note we consider symbolic dynamics on $\mathbb{Z}$. Let $A$ denote the set of symbols, and $\mathbb{Z}$ acts on $X:=A^\mathbb{Z}$ by translations, i.e. $mx(n)=x(n-m)$.

$X$ is endowed with the product topology of discrete topology on $A$, which makes it a topological dynamical system. The subsystems in $(X,translation)$ are \emph{shifts.} Denote the space of all shifts $\mathfrak{S}$. It is endowed with the Hausdorff topology.



\subsection{$r$-boundary}
For $F\subset \mathbb{Z}$ finite, define its \emph{$r$-boundary} $\partial_r F$:
\begin{align*}
	\partial_r F :=\{n\in \mathbb{Z}:B_r(n)\cap F\neq \emptyset, B_r(n)\cap F^c\neq\emptyset\}
\end{align*}
and its \emph{boundary ratio}
\begin{align*}
	\rho_r F:=\frac{|\partial_r F |}{|F|}
\end{align*}
say $F$ is \emph{ $(r,\varepsilon)$-invariant} if $\rho_r F\leq \varepsilon$.




\section{Quasi-tiling shift}

A tile set is a 





\end{document}
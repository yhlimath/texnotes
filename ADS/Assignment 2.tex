\documentclass{article}
\usepackage{amsmath}
\usepackage{mathrsfs}
\usepackage{amssymb}
\usepackage{dcolumn}
\usepackage{graphicx}
\usepackage{fancybox}
\usepackage{amsmath}
\usepackage{amsthm}
\usepackage{amsrefs}
\usepackage{tikz-cd}

\usepackage{CJK}


\newtheorem{definition}{Definition}
\newtheorem{theorem}{Theorem}
\newtheorem{proposition}{Proposition}
\newtheorem{lemma}{Lemma}
\newtheorem{corollary}{Corollary}
\newtheorem{example}{Example}
\newtheorem{remark}{Remark}

\renewcommand{\today}{\number\year.\number\month.\number\day}






\begin{document}


\title{Assignment 2}
\author{Li Yueheng s2306706}

\maketitle
In this document $F(x)$ is the vector field of the dynamical system we discuss.
\section{}

We first calculate the equilibriums of the system. Set
\begin{center}
$
	\begin{cases}
		\dot{x}=y-x^3=0 \\
		\dot{y}=-x-y^3=0
	\end{cases}
$
\end{center}
trivial calculation shows that there is a unique equilibrium at $(0,0)$.

Define $V=x^2+y^2$. Obviously this function is positive everywhere except at $0$, which is the unique equilibrium, and $V(0)=0$. One the other hand, 
\begin{align*}
	\langle \nabla V,F(x)\rangle =(2x,2y)\cdot(y-x^3,-x-y^3)^T=-2(x^4+y^4)\leq 0
\end{align*}
and the inequality is strict everywhere but at $(0,0)$. Thus $V$ is a Lyapunov function. 

Should the system has a closed orbit $\mathcal{O}$ except fixed point, then let $x_0 \in \mathcal{O}$, there exists $T>0$ such that $\phi_T(x_0)=x_0$. 

Set $x(t)$ to be the integral curve that starts at $x_0$. 
\begin{align*}
	\frac{d}{dt}V(x(t))=\langle \nabla V(x(t)),F(x(t))\rangle \leq0
\end{align*}
Since $x(t)$ does not pass $0$ which is the only point where the above inequality isn't strict, we may conclude that $V(x(t))$ decreases strictly with respect to $t$. That leads to a contradiction:
\begin{align*}
	V(x_0)=V(x(0))>V(x(T))=V(x_0)
\end{align*}
Thus there is no closed orbit.




\section{}
\subsection{}
Set
\begin{center}
$
	\begin{cases}
	\dot{u}=-u+av+u^2v=0 \\
	\dot{v}=b-av-u^2v=0
\end{cases}
$
\end{center}
the unique equilibrium is 
\begin{center}
$
\begin{cases}
	u^*=b \\
	v^*=\frac{b}{a+b^2}
\end{cases}
$
\end{center}

Set $x^*:=(u^*,v^*)$. Compute the Jacobian $JF(x^*)$:
\begin{equation*}
	JF(x^*)=
	\left(
	\begin{array}{cc}
	\partial_u F_u & \partial_v F_u \\
	\partial_u F_v & \partial_v F_v
	\end{array}
	\right)(x^*)
	=
	\left(
	\begin{array}{cc}
		-1+2u^{*} v^{*} & a+u^{*2} \\
		-2u^{*} v^{*} & -a-u^{*2}
	\end{array}
	\right)
	=
	\left(
	\begin{array}{cc}
		\frac{-a+b^2}{a+b^2} & a+b^2 \\
		\frac{-2b^2}{a+b^2} & -a-b^2
	\end{array}
	\right)
\end{equation*}

Notice that 
\begin{align*}
	&Tr(JF(x^*))=\frac{-a+b^2-(a+b^2)^2}{a+b^2} \\
	&Det(JF(x^*))=a+b^2>0
\end{align*}

Since the determinant is always positive, the equilibrium is stable iff $Tr(JF(x^*))>0$, and is unreachable, i.e. has no eigenvalue with non-negative real part, iff $Tr(JF(x^*))<0$. Thus,

\begin{itemize}
	\item $x^*$ is stable iff $-a+b^2-(a+b^2)^2<0$
	\item $x^*$ is unreachable (or totally unstable) iff $-a+b^2-(a+b^2)^2>0$
\end{itemize}

In addition, $x^*$ cannot be a saddle point. And solutions near it are spirals iff $Tr(JF(x^*))^2-4Det(JF(x^*))<0$.

\subsection{}
First we compute the outer normal vectors for the boundaries given. For $\partial S_i$ denote its outer normal vector as $n_i$. The computations are elementary, so detailed calculations are omitted.
\begin{align*}
	&n_1=(0,-1) \\
	&n_2=(1,1) \\
	&n_3=(0,1) \\
	&n_4=(-1,0)
\end{align*}
Then we estimate $\langle F(x),n_i \rangle$ on the boundaries.
\begin{align*}
	& F(u,v)=(-u+av+u^2v,b-av-u^2v)^T \\
	&\langle F(x),n_1\rangle =-\dot{v}=u-av-u^2v=-b<0 \quad\mbox{since when $x\in \partial S_1$, $v\equiv 0$.} \\
	&\langle F(x),n_2\rangle =\dot{u}+\dot{v}=b-u\leq0  \quad\mbox{since when $x\in\partial S_2$, $u\geq b$.} \\
	&\langle F(x),n_3\rangle =\dot{v}=b-av-u^2v=-u^2v\leq 0 \quad\mbox{since when $x\in \partial S_3$, $v\equiv \frac{b}{a}$} \\
	&\langle F(x),n_4\rangle=-\dot{u}=u-av-u^2v=-av\leq 0 \quad\mbox{since when $x\in\partial S_4$, $u\equiv 0$}
\end{align*}
Hence $S$ is a trapping region.

\subsection{}
The only equilibrium of the system is $x^*$, which lies within $S$ but is totally unstable. Thus we choose a small ball $B_r(x^*)$: for any $x\in \partial B_r(x^*)$,
\begin{align}\label{eq1}
	\langle F(x),n_x\rangle=\langle JF(x^*)(x-x^*),x-x^*\rangle+o(\lVert x-x^*\rVert^2)
\end{align}
Since $x^*$ is totally unstable, from the theory of linear ODE we know for $B_r(x^*)$ sufficiently small, the integral curve of the linear system $\dot{x}=JF(x^*)(x-x^*)$ that originates at $x$ must bent strictly outward $B_r(x^*)$, that is $$\langle JF(x^*)(x-x^*),x-x^*\rangle>0$$ for any $x\in\partial B_r(x^*)$. 
Notice that $$f(x-x^*):=\langle JF(x^*)(x-x^*),x-x^*\rangle$$ is a positive continuous function defined on a compact set $\partial B_r(x^*)$, so it achieves a positive minimum $m\lVert x-x^*\rVert^2$ on it. Notice also that $f$ is a quadratic homogenous function, which means the above conclusion holds for $B_r(x^*)$ of any radius, that is, the constant $m$ does not change with respect to $r$.

Therefore we may conclude that for $r$ sufficiently small, the expression \ref{eq1} is positive:
\begin{align*}
	\langle F(x),n_x\rangle &=\langle JF(x^*)(x-x^*),x-x^*\rangle+o(\lVert x-x^*\rVert^2)\\
	&\geq m\lVert x-x^*\rVert^2 -o\lVert x-x^*\rVert^2 \geq 0
\end{align*}

Our final conclusion is $S-B_r(x^*)$ is positively invariant, as we've verified $\langle F(x),n\rangle\leq0$ on its boundary, where $n$ is its outer normal vector at $x$. By Poincar\'{e}-Bendixson theorem, $S-B_r(x^*)$ contains a closed orbit.




\end{document}
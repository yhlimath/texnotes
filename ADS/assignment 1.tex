\documentclass{article}
\usepackage{amsmath}
\usepackage{mathrsfs}
\usepackage{amssymb}
\usepackage{dcolumn}
\usepackage{graphicx}
\usepackage{fancybox}
\usepackage{amsmath}
\usepackage{amsthm}
\usepackage{amsrefs}
\usepackage{tikz-cd}

\usepackage{CJK}


\newtheorem{definition}{Definition}
\newtheorem{theorem}{Theorem}
\newtheorem{proposition}{Proposition}
\newtheorem{lemma}{Lemma}
\newtheorem{corollary}{Corollary}
\newtheorem{example}{Example}
\newtheorem{remark}{Remark}

\renewcommand{\today}{\number\year.\number\month.\number\day}






\begin{document}


\title{Assignment 1}
\author{Li Yueheng s2306706}
\date{\today}
\maketitle


\section*{Problem 1}
\subsection*{1}
$g'(x)=-2\alpha(x-\eta)$ decreases monotonously. therefore $g$ achieves a unique maximum $r_0$ at $x_{max}=\eta$.


\subsection*{2}
Set $\dot{x}=0$ to find three solutions $\{0,\eta+\sqrt{\frac{r_0}{\alpha}},\eta-\sqrt{\frac{r_0}{\alpha}}\}$. Among which the first two are distinct non-negative solutions. They are the steady states of the system.

Near $0$, we have $ x\dot{x} =x^2 g(x) \leq 0$. 

Near $\eta+\sqrt{\frac{r_0}{\alpha}}$, we have $(x-(\eta+\sqrt{\frac{r_0}{\alpha}}))\dot{x}\leq 0$.

So, the system will return to these two steady states under perturbation. Therefore these steady states are stable.


\subsection*{3}
Under the assumption $\eta-\sqrt{\frac{r_0}{\alpha}}>0$, the arguments in the previous problem still hold. So $0$ and $\eta+\sqrt{\frac{r_0}{\alpha}}$ are stable steady states. 

While near $\eta-\sqrt{\frac{r_0}{\alpha}}$, $(x-(\eta-\sqrt{\frac{r_0}{\alpha}}))\dot{x}\geq 0$. If $x\neq \eta-\sqrt{\frac{r_0}{\alpha}}$, the inequality is strict. Therefore a perturbation will be magnified, and any perturbation of the system will result in it the system leaving a neighborhood of the steady state. This neighborhood is irrelevant with the perturbation, which means $\eta-\sqrt{\frac{r_0}{\alpha}}$ is unstable.


\subsection*{4}
$x_c:=\eta-\sqrt{\frac{r_0}{\alpha}}$ is a critical value. When the initial population is smaller than $x_c$, the horde can no longer sustain itself and gradually evolves toward extinction (the steady state $0$). While when the initial population is larger than $x_c$, the horde expands toward its maximum size $\eta+\sqrt{\frac{r_0}{\alpha}}$.

$0$ stands for extinction while $\eta+\sqrt{\frac{r_0}{\alpha}}$  stands for ecological equilibrium.


\section*{Problem 2}



\subsection*{1}
Solutions to the homogenous equation take the form $x_{hom}(t)=C_1e^t+C_2e^{-t}$, and a solution to the equation is $x_p(t)=-\frac{1}{2}cost$. Therefore a general solution would take the form
\begin{align*}
	x(t)=\frac{1}{2}(x(0)+\dot{x}(0)+\frac{1}{2})e^t+ \frac{1}{2}(x(0)-\dot{x}(0)+\frac{1}{2}) e^{-t}-\frac{1}{2}cost
\end{align*}



\subsection*{2}
\begin{align*}
	y:=\dot{x} \\
	t(\tau):=\tau 
\end{align*}

then the system has an autonomous form 

	
	\begin{equation*}
	\frac{d}{d\tau}\left[
	\begin{array}{c}
		 x\\
		 y\\
		 t
	\end{array}
	\right]=
	\left[
	\begin{array}{c}
		 y\\
		 x+cost\\
		 1
	\end{array}
	\right]
\end{equation*}



\subsection*{3}
Given initial condition $(\sigma_x,\sigma_y)\in \Sigma=\mathbb{R}^2$, consider its image under the flow $\phi_{2\pi}$. Define the $Poincar\acute{e}$ map $P$ as 
\begin{align*}
	P:\Sigma\times\{0\}&\rightarrow\Sigma\times\{0\} \\
	(\sigma_x,\sigma_y,0)&\mapsto (\phi_{2\pi}(\sigma_x,\sigma_y),0)
\end{align*}
where
\begin{align*}
	&p_x\circ\phi_{2\pi}(\sigma_x,\sigma_y)=\frac{1}{2}(\sigma_x+\sigma_y+\frac{1}{2})e^{2\pi}+ \frac{1}{2}(\sigma_x-\sigma_y+\frac{1}{2}) e^{-2\pi}-\frac{1}{2} \\
	&p_y\circ\phi_{2\pi}(\sigma_x,\sigma_y)=\frac{1}{2}(\sigma_x+\sigma_y+\frac{1}{2})e^{2\pi}- \frac{1}{2}(\sigma_x-\sigma_y+\frac{1}{2}) e^{-2\pi}
\end{align*}
where $p_i$ are projections.



\end{document}
%\documentclass[handout]{beamer}
\documentclass{beamer}
%
% See the beamer documentation at:
% http://mirrors.ctan.org/macros/latex/contrib/beamer/doc/beameruserguide.pdf
% for details of all the package options if you are interested.
%
%
% This project is just a starting point for you to make your own
% set of slides. Change it as you wish. Keeping it handy will probably be 
% useful for the future too.
%
\mode<presentation> 
{
  \usetheme{default}      % or try Boadilla, Darmstadt, Madrid, Warsaw, ...
  \usecolortheme{default} % or try albatross, beaver, crane, ...
  \usefonttheme{default}  % or try serif, structurebold, ...
  \setbeamertemplate{navigation symbols}{} % This turns off the navigation bar. Some people like this.
  \setbeamertemplate{caption}[numbered]
}

\usepackage[utf8]{inputenc}
\usepackage[UKenglish]{babel}
\usepackage{csquotes}
\usepackage{adjustbox}

% ========= Bibliography =========
% These lines load the `biblatex' package
% and read in the list of references from
% References.bib - take a look.
%
% To generate References.bib, I recommend https://www.mybib.com/
% rather than trying to write the .bib file yourself.
%
% style=verbose because we want full citations where they are 
% made in the presentation so that the audience doesn't
% have to wait until the end to know what the references are.
%
\usepackage[style=verbose]{biblatex}
\addbibresource{References.bib}
% ================================

\title[Short title]{Full title of presentation}
\author{Your name, s0000000}
\institute{The University of Edinburgh}
\date{\today}

\begin{document}

% ==============================
\begin{frame}
  \titlepage
\end{frame}
% ==============================


% ==============================
\begin{frame}
\frametitle{Outline}
\tableofcontents
\end{frame}
% ==============================


\section{Section heading}

% ==============================
\begin{frame}{First heading}

\begin{itemize}
  \item Item 1\pause
  \item Item 2\pause
  \item Item 3\pause
\end{itemize}

\end{frame}
% ==============================


% ==============================
\begin{frame}{Second Heading}

\begin{enumerate}
\item Item 1
\item Item 2
\item Item 3
\end{enumerate}

\begin{block}{Block name}
Some content \footcite{bickerton_2021_honours} % We can use \footcite so that the citation appears on the same slide and the audience doesn't have to wait until the end to see a list of references.
\end{block}

\begin{example}
Some content
\end{example}

\end{frame}
% ==============================


\subsection{Subsection name}

% ==============================
\begin{frame}{Third heading}

\begin{enumerate}
\item Item 1
\item Item 2
\item Item 3
\end{enumerate}

\end{frame}
% ==============================


% ==============================
\begin{frame}

\begin{figure}
    \adjincludegraphics[center, width=10cm]{knot.pdf} % This uses the adjustbox package to centre the image - the code is slightly neater than the alternatives
    \caption{A Celtic knot}
    \label{fig:knot}
\end{figure}

\end{frame}
% ==============================


% ==============================
\begin{frame}
\printbibliography % This command prints the cited references.
\end{frame}
% ==============================

\end{document}
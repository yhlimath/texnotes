\documentclass{article}
\usepackage{amsmath}
\usepackage{mathrsfs}
\usepackage{amssymb}
\usepackage{dcolumn}
\usepackage{graphicx}
\usepackage{fancybox}
\usepackage{amsmath}
\usepackage{amsthm}
\usepackage{amsrefs}
\usepackage{tikz-cd}

\usepackage{CJK}


\newtheorem{definition}{Definition}
\newtheorem{theorem}{Theorem}
\newtheorem{proposition}{Proposition}
\newtheorem{lemma}{Lemma}
\newtheorem{corollary}{Corollary}
\newtheorem{example}{Example}
\newtheorem{remark}{Remark}

\renewcommand{\today}{\number\year.\number\month.\number\day}






\begin{document}


\title{Assignment 2}
\author{Li Yueheng s2306706}

\maketitle

\section*{Workshop2}
\subsection*{Problem 5}


$\forall z \in A_{r,R}(z_0)$, suppose $z=z_0+le^{i\theta}$. Define $d:=\min\{{\frac{1}{2}(l-r),\frac{1}{2}(R-l) }\}$. We claim that the open neighborhood $B_d(z)$ of $z$ that lies within $A_{r,R}(z_0)$.

Proof of the claim: $\forall w\in B_{d}(z)$, by the triangle inequality we have
\begin{align*}
	r< l-d \leq ||z_0-z|-|z-w|| \leq|z_0-w|\leq |z_0-z|+|z-w|\leq l+d <R.
\end{align*}
Therefore every point in $B_d(z)$ lies in $A_{r,R}(z_0)$.

Therefore $A_{r,R}(z_0)$ is open.

\subsection*{Problem 6}
Notice that $\mathbb{Q}\times i\mathbb{Q}$ is a countable dense set in $\mathbb{C}$ due to the following reasons:

1.$Q$ is dense in $\mathbb{R}$, so every complex number can be approximated by a complex number in $\mathbb{Q}\times i\mathbb{Q}$ in the sense that both its real and imaginary parts are approximated to any accuracy.

2. the inequality $|z|\leq |Rez|+|Imz|$ shows the approximation in 1. is also an approximation in the sense of the metric topology on the complex plane.

3.$\mathbb{Q}\times i\mathbb{Q}$ is a countable set, which is due to the diagonal rule of Cantor and the countability of $\mathbb{Q}$ (which is also due to the diagonal rule)


\section*{Workshop 3}
\subsection*{Problem 6}


\subsubsection*{a}
\begin{align*}
	\frac{g_1(z+h)-g_1(z)}{h}=\frac{\overline{f(\overline{z+h})}-\overline{f(\overline{z})}}{h}=\overline{\frac{f(\overline{z+h})-f(\overline{z})}{\overline{h}}}\overset{\Vert h \Vert\rightarrow 0}{\longrightarrow}\overline{f'(\overline{z})}.
\end{align*}
So $g_1$ is differentiable everywhere, which means it is holomorphic.

\subsubsection*{b}
Cauchy-Riemann equations of $f=u+iv$ and $g_2=u-iv$ give
\begin{align*}
	&\partial_x u=\partial_y v \\
	&\partial_y u=-\partial_x v \\
	&\partial_x u=-\partial_y v \\
	&\partial_y u=\partial_x v.
\end{align*}
Thus 
\begin{align*}
	\partial_y v=-\partial_y v =0 \\
	\partial_x v=-\partial_x v=0.
\end{align*}
Which means $v$ is a constant. A constant is naturally holomorphic, so $u=f-iv$ is also holomorphic because of the algebraic properties of holomorphic functions. A real holomorphic function is a constant, so $u$ is also a constant. Together we have shown $f=u+iv$ is a constant.


\subsubsection*{c}
Cauchy-Riemann equations of $f=u+iv$ and $g_3(x+iy)=u(x-iy)+iv(x-iy)$ give
\begin{align*}
	&\partial_x u=\partial_y v \\
	&\partial_y u=-\partial_x v \\
	&\partial_x u=-\partial_y v \\
	&\partial_y u=\partial_x v.
\end{align*}
and the remaining prove is similar to part b.

\subsection*{Problem 7}
Should $f=u+iv$ be a holomorphic function, Cauchy-Riemann equations must be satisfied. Therefore
\begin{align*}
	&\partial_y v=\partial_x u=by^2+2cxy+3dx^2 \\
	&\partial_x v=-\partial_y u=-3ay^2-2bxy-cx^2.
\end{align*}
integration gives
\begin{align*}
	v=\frac{1}{3}by^3+cxy^2+3dx^2y+g(x)=-3axy^2-bx^2y-\frac{1}{3}cx^3+h(y)
\end{align*}
compare these two expressions, the coefficients of the polynomial must satisfy

\begin{align*}
	\begin{cases}
		c=-3a \\
		3d=-b.
	\end{cases}
\end{align*}

This condition is also sufficient: when the polynomial satisfies this condition, its harmonic conjugate $v$ can be constructed as discussed above. Both the polynomial and its conjugate are smooth functions, and $f=u+iv$ satisfies Cauchy-Riemann equations. Thus $f$ is holomorphic.
\end{document}
\documentclass{article}
\usepackage{amsmath}
\usepackage{mathrsfs}
\usepackage{amssymb}
\usepackage{dcolumn}
\usepackage{graphicx}
\usepackage{fancybox}
\usepackage{amsmath}
\usepackage{amsthm}
\usepackage{amsrefs}
\usepackage{tikz-cd}

\usepackage{CJK}


\newtheorem{definition}{Definition}
\newtheorem{theorem}{Theorem}
\newtheorem{proposition}{Proposition}
\newtheorem{lemma}{Lemma}
\newtheorem{corollary}{Corollary}
\newtheorem{example}{Example}
\newtheorem{remark}{Remark}

\renewcommand{\today}{\number\year.\number\month.\number\day}






\begin{document}


\title{Assignment 2}
\author{Li Yueheng s2306706}

\maketitle


\section*{Problem 5}


$\forall z \in A_{r,R}(z_0)$, suppose $z=z_0+le^{i\theta}$. Define $d:=\min\{{\frac{1}{2}(l-r),\frac{1}{2}(R-l) }\}$. We claim that the open neighborhood $B_d(z)$ of $z$ that lies within $A_{r,R}(z_0)$.

Proof of the claim: $\forall w\in B_{d}(z)$, by the triangle inequality we have
\begin{align*}
	r< l-d \leq ||z_0-z|-|z-w|| \leq|z_0-w|\leq |z_0-z|+|z-w|\leq l+d <R
\end{align*}
Therefore every point in $B_d(z)$ lies in $A_{r,R}(z_0)$.

Therefore $A_{r,R}(z_0)$ is open.

\section*{Problem 6}
Notice that $\mathbb{Q}\times i\mathbb{Q}$ is a countable dense set in $\mathbb{C}$ due to the following reasons:

1.$Q$ is dense in $\mathbb{R}$, so every complex number can be approximated by a complex number in $\mathbb{Q}\times i\mathbb{Q}$ in the sense that both its real and imaginary parts are approximated to any accuracy.

2. the inequality $|z|\leq |Rez|+|Imz|$ shows the approximation in 1. is also an approximation in the sense of the metric topology on the complex plane.

3.$\mathbb{Q}\times i\mathbb{Q}$ is a countable set, which is due to the diagonal rule of Cantor and the countability of $\mathbb{Q}$ (which is also due to the diagonal rule)
\end{document}
\documentclass{article}
\usepackage{amsmath}
\usepackage{mathrsfs}
\usepackage{amssymb}
\usepackage{dcolumn}
\usepackage{graphicx}
\usepackage{fancybox}
\usepackage{amsmath}
\usepackage{amsthm}
\usepackage{amsrefs}
\usepackage{tikz-cd}

\usepackage{CJK}


\newtheorem{definition}{Definition}
\newtheorem{theorem}{Theorem}
\newtheorem{proposition}{Proposition}
\newtheorem{lemma}{Lemma}
\newtheorem{corollary}{Corollary}
\newtheorem{example}{Example}
\newtheorem{remark}{Remark}

\renewcommand{\today}{\number\year.\number\month.\number\day}






\begin{document}


\title{Assignment 3}
\author{Li Yueheng s2306706}

\maketitle


\section*{Question 5}

Notice that 
\begin{align*}
	(1+z^2)^{1/2}&=((z+i)(z-i))^{1/2}=\exp\{\frac{1}{2}(\log(z+i)+\log(z-i))\} \\
	&=:\exp\{\frac{1}{2}(g(z)+g(z))\}
\end{align*}

Since $g$ and $h$ are compositions of the logarithmic multifunction and translations, among them translations are obviously holomorphic on $D_1(0)$, it suffices to choose appropriate branches of logarithmic multifunction so that their brunch cuts do not intersect the images of $D_1(0)$ under translations. In doing so, by the composition law of holomorphic functions, the original multifunction would be holomorphic on $D_1(0)$.

Notice that $i+D_1(0)=D_1(i)$ and $D_1(0)-i=D_1(-i)$. For $g$ to be holomorphic, we may define $g(z)=Log_{1,0}(z+i)$. While for $h$, define $h(z)=Log_{1,0}(z-i)$.

\section*{Question 6}
Notice that 
\begin{align*}
	(1+z^2)^{1/2}=(z^2(1+z^{-2}))^{1/2}=(z^2)^{1/2}(1+z^{-2})^{1/2}
\end{align*}
on $\mathbb{C}-\overline{D_1(0)}$.

Simply choose $(z^2)^{1/2}=z$, which is holomorphic on the whole complex plane. 

As for $(1+z^{-2})^{1/2}=\exp\{\frac{1}{2}\log(1+z^{-2})\}$, choose $(1+z^{-2})^{1/2}=\exp\{\frac{1}{2}Log_{3,0}(1+z^{-2})\}$. 

Since $(1+z^{-2})(\mathbb{C}-\overline{D_1(0)})\subset D_2(0)$, while the branch cut of $Log_{3,0}$ lies outside $D_3(0)$, they will not intersect. Thus the branch we chose is holomorphic on $\mathbb{C}-\overline{D_1(0)}$.






\end{document}
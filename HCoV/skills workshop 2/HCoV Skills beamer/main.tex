%\documentclass[handout]{beamer}
\documentclass{beamer}
%
% See the beamer documentation at:
% http://mirrors.ctan.org/macros/latex/contrib/beamer/doc/beameruserguide.pdf
% for details of all the package options if you are interested.
%
%
% This project is just a starting point for you to make your own
% set of slides. Change it as you wish. Keeping it handy will probably be 
% useful for the future too.
%
\mode<presentation> 
{
  \usetheme{Madrid}      % or try Boadilla, Darmstadt, Madrid, Warsaw, ...
  \usecolortheme{default} % or try albatross, beaver, crane, ...
  \usefonttheme{default}  % or try serif, structurebold, ...
  \setbeamertemplate{navigation symbols}{} % This turns off the navigation bar. Some people like this.
  \setbeamertemplate{caption}[numbered]
}

\usepackage[utf8]{inputenc}
\usepackage[UKenglish]{babel}
\usepackage{csquotes}
\usepackage{adjustbox}

% ========= Bibliography =========
% These lines load the `biblatex' package
% and read in the list of references from
% References.bib - take a look.
%
% To generate References.bib, I recommend https://www.mybib.com/
% rather than trying to write the .bib file yourself.
%
% style=verbose because we want full citations where they are 
% made in the presentation so that the audience doesn't
% have to wait until the end to know what the references are.
%

\usepackage[style=verbose]{biblatex}
\addbibresource{References.bib}

\newtheorem{remark}{Remark}
\newtheorem{proposition}{Proposition}
% ================================

\title[the Complex Logarithm]{A Review of the Complex Logarithm}
\author{Li Yueheng, s2306706}
\institute{The University of Edinburgh}
\date{\today}

\begin{document}

% ==============================
\begin{frame}
  \titlepage
\end{frame}
% ==============================


% ==============================
\begin{frame}
\frametitle{Outline}
\tableofcontents
\end{frame}
% ==============================


\section{Definition and basic properties}

% ==============================
\begin{frame}{Definition of the complex logarithm}
\begin{definition}[1.7.1 on p.16, Gratwick]
		For $z\in \mathbb{C}-\{0\}$, define its \emph{logarithm} by
	\begin{align*}
		log(z):=\{w\in\mathbb{C}:exp(w)=z\}
	\end{align*}
\end{definition}\pause

\begin{remark}
	The complex logarithm is a \emph{multivalued function}. We call any element of the set $log(z)$ a \emph{logarithm} of $z$.
\end{remark}



\end{frame}
% ==============================


% ==============================
\begin{frame}{Basic properties of the complex logarithm}

\begin{proposition}[1.7.3 on p.16, Gratwick]
	Let $z,w\in\mathbb{C}-\{0\}$, and $z=re^{i\theta}$ in exponential form. Then
	\begin{enumerate}
		\item $\log(z)=\{\ln r+i\theta+i2\pi k:k\in\mathbb{Z}\}$\quad (Depiction of the set $\log(z)$)\pause
		\item $\log(zw)=\log(z)+\log(w)$\quad(Law of multiplication)\pause
		\item $\log(1/z)=-\log(z)$\quad (Law of reciprocal)\pause
	\end{enumerate}
\end{proposition}
	
\begin{remark}
	These statements should be understood as equalities between \emph{sets}.
\end{remark}


\end{frame}
% ==============================
\begin{frame}{Sketch of proof}
	1.
	Let $w=a+ib$ and solve the equation
		$$exp(w)=exp(a+ib)=re^{i\theta}=z$$
	to find all $w\in log(z)$.\pause
\vspace{3ex}

	2.
	The law of multiplication follows directly from our depiction of the logarithm.\pause
\vspace{3ex}

	3.
	Notice that $log(1)=\{i2\pi k:k\in\mathbb{Z}\}$ and use the law of multiplication.
	
\end{frame}


\section{Branches of the logarithm and branch cuts}

% ==============================

% ==============================


% ==============================
\begin{frame}{Brunches of the logarithm}



\begin{definition}[1.7.9 on p.17, Gratwick]
	A \emph{brunch} of the logarithm is defined as a function $Log_{\phi}:\mathbb{C}-\{0\}\to\mathbb{C}$,
	\begin{align*}
		Log_{\phi}(z):=\ln |z|+iArg_{\phi}(z)
	\end{align*}
	where $Arg_{\phi}(z)\in arg(z)\cap (\phi,\phi+2\pi]$ is a properly defined real number.
\end{definition}\pause

\begin{example}
	\emph{The principal brunch} of the logarithm is the function $Log:\mathbb{C}-\{0\}\to\mathbb{C}$ defined as 
		\begin{align*}
			Log(z):=Log_{-\pi}(z)=\ln |z|+iArg(z)
		\end{align*}
\end{example}

\end{frame}

% ==============================

\begin{frame}{Brunch cuts and the cut plane}

\begin{definition}[1.7.7 on p.17, Gratwick]
	A \emph{brunch cut} $L$ is a subset removed from the complex plane $\mathbb{C}$ so that a holomorphic branch of a multivalued function may be defined on $D=\mathbb{C}-L$, the cut plane.
\end{definition}\pause


\begin{remark}
	\emph{Half-lines} are an important class of branch cuts, so we introduce the following notation: For $z_0\in\mathbb{C}$ and $\phi\in\mathbb{R}$, define
	\begin{align*}
		L_{z_0,\phi}=\{z\in\mathbb{C}:z=z_0+re^{i\phi},r\geq 0\}
	\end{align*}
	and define its cut plane $D_{z_0,\phi}:=\mathbb{C}-L_{z_0,\phi}$. In particular, $D_\phi:=D_{0,\phi}$.
\end{remark}
\end{frame}
% ==============================


\section{Holomorphicity gained and continuity lost}
\begin{frame}{Holomorphicity gained}

	\begin{theorem}[1.7.10 on p.18, Gratwick]
	A brunch of the logarithm $Log_{\phi}$ is holomorphic on the cut plane $D_\phi$, and $Log_{\phi}'(z)=1/z$ in $D_\phi$.
	\end{theorem}\pause
	
	\begin{proof}
	A sketch:
		\begin{enumerate}
			\item 1.Verify that $Log_{\phi}(z)=\ln(|z|)+iArg_{\phi}(z)$ has continuously differentiable real and imaginary parts on $D_\phi$.\pause
			\item 2.Check that the Cauchy-Riemann equations are satisfied on $D_\phi$. These two steps combined show that $Log_\phi$ is holomorphic on $D_\phi$.\pause
			\item 3.Differentiate $z=exp(Log_\phi(z))$ using the chain rule to compute the derivative of $Log_{\phi}$ on $D_\phi$.
		\end{enumerate}
	\end{proof}
	
	
	
	
\end{frame}

\begin{frame}{Continuity lost}
	Since $Arg_\phi(z)$ takes a jump when $z$ crosses $L_{0,\phi}$,  it isn't continuous there, and hence isn't $Log_\phi(z)=\ln(|z|)+iArg_\phi(z)$.\pause
	
	\begin{example}
	 Notice that \begin{align*}
	 	Arg_0(1)=2\pi
	 \end{align*} while 
	 \begin{align*}
	 	Arg_0(e^{i\varepsilon})=\varepsilon
	 \end{align*}
	 Thus $2\pi=Arg_0(1)=Arg_0(\lim_{\varepsilon\to 0}e^{i\varepsilon})\neq\lim_{\varepsilon\to0}Arg_0(e^{i\varepsilon})=0$
\end{example}
\end{frame}



\begin{frame}{To conclude}
	\begin{enumerate}
		\item The complex logarithm is a multivalued function.\pause
		\item The complex logarithm "inherits" some properties of the real one, but just formally!\pause
		\item By cutting the complex plane, we may define branches of the logarithm, which are \emph{functions}.\pause
		\item Branches lose continuity on the branch cuts, but gain holomorphicity elsewhere.\pause
	\end{enumerate}
\end{frame}



% ==============================
\begin{frame}{References}
\begin{thebibliography}
	\bibitem{1}.Gratwick, R. (2022). Honours Complex Variables Lecture Notes 2021–2022. Accessed February 13th, 2022.
\end{thebibliography}
\end{frame}
% ==============================

\end{document}
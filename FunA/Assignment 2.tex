\documentclass{article}
\usepackage{amsmath}
\usepackage{mathrsfs}
\usepackage{amssymb}
\usepackage{dcolumn}
\usepackage{graphicx}
\usepackage{fancybox}
\usepackage{amsmath}
\usepackage{amsthm}
\usepackage{amsrefs}
\usepackage{tikz-cd}

\usepackage{CJK}


\newtheorem{definition}{Definition}
\newtheorem{theorem}{Theorem}
\newtheorem{proposition}{Proposition}
\newtheorem{lemma}{Lemma}
\newtheorem{corollary}{Corollary}
\newtheorem{example}{Example}
\newtheorem{remark}{Remark}

\renewcommand{\today}{\number\year.\number\month.\number\day}






\begin{document}


\title{Assignment 2}
\author{Li Yueheng s2306706}

\maketitle


\section*{Question 1}
\subsection*{a}
$$A:=\{(x_1,x_2):f(x_1)=f(x_2)\} $$
For any $y_1\in Y$, $y_2\in Y$ and $y_1\neq y_2$, since $f$ is onto, we may choose $x_1,x_2$ such that $f(x_i)=y_i$. Then $(x_1,x_2)\notin A$. Since $A$ is closed, there exists  an open neighborhood $W$ of $(x_1,x_2)$ such that $W\cap A=\emptyset$. From the definition of product topology on $X\times X$ we know there exists $U\times V$ such that 
\begin{enumerate}
	\item $U$ and $V$ are open neighborhoods of $x_1$ and $x_2$.
	\item $U\times V \subset W \subset A^c$.
\end{enumerate}
Since $f$ is open, $f(U)$ and $f(V)$ are open neighborhoods of $y_1$ and $y_2$. They cannot intersect, otherwise there exist $z\in U$ and $w\in V$ such that $f(w)=f(z)$, thus $(z,w)\in A$, but $(z,w)\in U\times V\subset W \subset A^c$, a contradiction. Hence $f(U)$ and $f(V)$ are disjoint open neighborhoods of $y_1$ and $y_2$, so $Y$ is Hausdorff.


\subsection*{b}
$$B:=\{x:f(x)=g(x)\}$$

For any $x\notin B$, $f(x)\neq g(x)$. Since $Y$ is Hausdorff, there exist disjoint open neighborhoods of $f(x)$ and $g(x)$, namely $U$ and $V$. Since $f$ and $g$ are continuous, $f^{-1}(U)$ and $g^{-1}(V)$ are open. Then since $W:=f^{-1}(U)\cap g^{-1}(V)$ is the intersection of two open sets, $W$ itself is also open. From $f(x)\in U$ and $g(x)\in V$ we find that $x\in W$. 

For any $w\in W$, we have $f(w)\in U$ and $g(w)\in V$ while $U\cap V=\emptyset$. Thus $f(w)\neq g(w)$, so $w\notin B$. Therefore $W\cap B=\emptyset$ and $x\in W$. Hence $B^c$ is open, or equivalently, $B$ is closed. 





\section*{Question 2}
\subsection*{a}
Suppose $f:[0,1]\to (0,1)$ is a homeomorphism, then $\tilde{f}:=f|_{(0,1]}$ is a homeomorphism between $(0,1]$ and $(0,1)-\{f(0) \}$. But while $ (0,1]$ is connected (because it is path connected), the other is not (because $(0,1)=(0,f(0))\cup (f(0),1)$), these two spaces cannot be homeomorphic since connectedness is invariant under homeomorphism. So, $f$ must not exist.

\begin{lemma}[I don't know if I need to prove this lemma or not]
Connectedness is invariant under homeomorphism.
\begin{proof}
		Let $f:X\to Y$ be a homeomorphism. Suppose $X$ is connected while $Y$ is not, then $Y=U\sqcup V$ where $U$ and $V$ are disjoint non-empty open sets. Since $f$ is a continuous surjection while $U$ and $V$ are non-empty, $f^{-1}(U)$ and $f^{-1}(V)$ are non-empty open sets in $X$. Since $U\cap V=\emptyset$, they are disjoint. Since $U\sqcup V=Y$, we have $f^{-1}(U)\sqcup f^{-1}(V)=X$, which contradicts the fact that $X$ is connected. 
		
		Suppose $Y$ is connected and $X$ isn't, the same argument applies since $f^{-1}$ is also a continuous bijection. In summary, $X$ and $Y$ must have same connectedness should they be homeomorphic.
\end{proof}
\end{lemma}


\subsection*{b}
Suppose $f:\mathbb{R}^2\to \mathbb{R}$ is a homeomorphism. Then $\tilde{f}:=f|_{x\neq 0}$ is a homeomorphism between the punctured plane and $\mathbb{R}-\{f(0)\}$. The punctured plane is path connected since we may connect any two points on the punctured plane with a spiral, and thus it is also connected. While on the other hand $\mathbb{R}-\{f(0)\}=(-\infty,f(0),)\cup (f(0),\infty)$ isn't connected.

 Hence $f$ must not exist.




\section*{Question 3}


Let $A$ and $B$ be two disjoint closed sets on the Sorgenfrey line. 

For each $a\in A$, since $a\in B^c$ and $B$ is a closed set, $a$ has an open neighborhood that lies within $B^c$. Therefore, there exists an element of the topological base, $[a,a')$ such that $[a,a')\subset B^c$. Define an open neighborhood of $A$ as
\begin{align*}
	U:=\bigcup_{a\in A}[a,a') 
	\end{align*}
Similarly define an open neighborhood $V$ for $B$:
\begin{align*}
	V:=\bigcup_{b\in B}[b,b')
\end{align*}
where each $[b,b')$ is an open neighborhood of $b$ that does not intersect $A$.

For any pair $a\in A, b\in B$, without loss of generality assume $a<b$. 
Since $[a,a') \subset B^c$, we have $[a,a')\cap [b,b')=\emptyset$. Thus $U\cap V=\emptyset$. Hence the Sorgenfrey line is $T_4$.




\section*{Question 4}
\subsection*{a}
Let $l_0\in l^\infty$ and $\Vert l_0 \rVert_\infty=1$. $l_{n+1}:=Tl_n$. $r_n:=\lVert l_n\rVert_\infty$. Then
\begin{align*}
	\lVert T^k  \rVert = \max_{\lVert l_0\rVert_\infty=1}\lVert T^k l_0 \rVert = \max_{\lVert l_0 \rVert_\infty=1}r_k
\end{align*}
Suppose 
\begin{align*}
	l_n=(a_1,a_2,a_3,\dots) 
\end{align*}
then
\begin{align*}
	&l_{n+1}=(a_1+a_2,a_1,a_2,\dots) \\
	&l_{n+2}=(a_1+a_2+a_1,a_1+a_2,a_1,\dots)
\end{align*}
Since every term of $l_n$ present in $l_{n+1}$, we have 
\begin{align}
	r_{n+1}= \max\{r_n,|a_1+a_2|\}\leq 2r_{n} \label{ineq}
\end{align}
Since all but the first term of $l_{n+2}$ present in $l_{n+1}$, and the first term of $l_{n+2}$ can be controlled by 
\begin{align*}
	|a_1+a_2+a_1|\leq |a_1|+|a_1+a_2|\leq \lVert l_n\rVert +\lVert l_{n+1}\rVert=r_n+r_{n+1}
\end{align*}
thus we have 
\begin{align}
	r_{n+2}\leq r_{n+1}+r_n \label{ineq2}
\end{align}
Since $r_0=1$, it follows from inequality \ref{ineq} that $r_1\leq 2$. 

With inequality \ref{ineq2}, it is easy to prove using induction that $r_n$ is always not greater than the $(n+1)th$ number in the Fibonacci sequence.

Notice that the above inequalities becomes equality when $l_0=(1,1,0,\dots)$ and $\lVert l_0\rVert=1$. 

Therefore by the general formula of the Fibonacci sequence, (I believe I could assume this formula is known? As it's a common exercise in first-year linear algebra courses)
$$\lVert T^k \rVert= \max_{\lVert l_0\rVert=1}r_k =\frac{1}{\sqrt{5}}((\frac{1+\sqrt{5}}{2} )^{k+2}-(\frac{1-\sqrt{5}}{2})^{k+2} )$$
and simple calculation yields
\begin{align*}
	r(T)=\lim_{k\to\infty} \lVert T^k\rVert^{1/k}=\frac{1+\sqrt{5}}{2}
\end{align*}

\subsection*{b}
Suppose $\lambda$ is an eigenvalue of $T$ and $v$ is its corresponding eigenvector with $\lVert v \rVert_\infty=1$. Then
\begin{align*}
	&|\lambda|=\frac{\lVert Tv\rVert_\infty}{\lVert v\rVert_\infty}\leq \max_{\lVert v\rVert_\infty=1}\frac{\lVert Tv\rVert_\infty}{\lVert v\rVert_\infty}=\lVert T\rVert \\
	&|\lambda |=(|\lambda|^k)^{1/k}=(\frac{\lVert T^k v\rVert_\infty}{\lVert v\rVert_\infty} )^{1/k}\leq \lVert T^k\rVert^{1/k}
\end{align*}
taking limits on both sides yield $|\lambda|\leq r(T)$.

\section*{Question 5}
Let $J=[a,b]$ be any bounded closed interval. Suppose $\{U_i\}_{i\in I}$ is an open cover of $J$ but does not have a finite subcover. Divide J by half into two closed intervals 
\begin{align*}
	J_{11}=[a,\frac{a+b}{2}] \quad  J_{12}=[\frac{a+b}{2},b]
\end{align*}

Then, at least one of these two intervals does not have a finite subcover. Again divide this interval without finite subcover by half and repeat to choose the one without finite subcover. Such process lasts endlessly and will result in a series of nested intervals, every one of them does not have a finite subcover. By the nested intervals theorem, there is a point $x_0$ that belongs to all the intervals. Since $\{U_i\}_{i\in I}$ covers $J$, there should be an open set $U_{i_0}$ such that $x_0\in U_{i_0}$. But as the lengths of the nested intervals tend to $0$, at least one of them lies within $U_{i_0}$. This contradicts the fact that none of these intervals has a finite subcover. Hence $\{U_{i}\}_{i\in I}$ must have a finite subcover for $J$.




\section*{Question 6}
Firstly, we show that the weak topology $\tau$ on $X$ satisfies the given universal property:

Suppose $g$ is continuous, since the weak topology makes all the $f_i$s continuous, it follows that $f_i\circ g$ is continuous.

Then suppose $f_i\circ g$ is continuous for all $i$s. Fix $i$, take any open set $U_i\in \tau_i$. it follows from the continuity of $f_i$ and $f_i\circ g$ that $f_i^{-1}(U_i)\in \tau $ and 
$g^{-1}(f^{-1}_i(U_i) )\in \tau_Z$. Thus sets of the form $f_i^{-1}(U_i)$ are pulled back by $g$ into $\tau_Z$.


Notice that
\begin{align*}
	\{f_{i}^{-1}(U_i):i\in I \mbox{ and } U_i \in \tau_i \}
\end{align*}
form a subbase of the weak topology $\tau$. It follows that this subbase is pulled back by $g$ into $\tau_Z$. Therefore every element of $\tau$ is pulled back by $g$ into $\tau_Z$, that is, $g$ is continuous.

\vspace{3ex}

For the other side of the proposition, assume $\tau'$ is a topology on $X$ that satisfies the universal property. Let $(Z,\tau_Z) =(X,\tau)$ and $g=$id$:X\to X$. Since $f_i\circ g$ is continuous, $g=$id is continuous, thus every open set of $\tau'$ is an open set of $\tau$, and $\tau' \subset \tau$.
\begin{center}
	\begin{tikzcd}
(X,\tau) \arrow[d, "g"] \arrow[rd,"f_i\circ g"] \\
(X,\tau') \arrow[r,"f_i"] & (X_i,\tau_i)
\end{tikzcd}
\end{center}

We then let $(Z,\tau_Z)=(X,\tau')$, which makes $g=$id continuous. Therefore $f_i\circ g$ is continuous according to the universal property. Hence for any $U_i \in \tau_i$, $f_i^{-1}(U_i) \in \tau'$. Since sets of the form $f_i^{-1}(U_i)$ form a subbase of the weak topology $\tau$, we may conclude that $\tau \subset \tau'$.

\begin{center}
	\begin{tikzcd}
(X,\tau') \arrow[d, "g"] \arrow[rd,"f_i\circ g"] \\
(X,\tau') \arrow[r,"f_i"] & (X_i,\tau_i)
\end{tikzcd}
\end{center}

The arguments above shows that $\tau=\tau'$, thereby shows that the universal property characterizes the weak topology.

\end{document}
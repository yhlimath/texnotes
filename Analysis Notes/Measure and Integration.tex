\documentclass{book}
\usepackage{amsmath}
\usepackage{mathrsfs}
\usepackage{amssymb}
\usepackage{dcolumn}
\usepackage{graphicx}
\usepackage{fancybox}
\usepackage{amsmath}
\usepackage{amsthm}
\usepackage{tikz-cd}

\usepackage{CJK}

\newtheorem{definition}{Definition}
\newtheorem{theorem}{Theorem}
\newtheorem{proposition}{Proposition}
\newtheorem{lemma}{Lemma}
\newtheorem{corollary}{Corollary}
\newtheorem{example}{Example}
\newtheorem{remark}{Remark}

\begin{document}
\begin{CJK}{UTF8}{gbsn}

\title{测度与积分}
\author{李岳恒}
\maketitle

\tableofcontents
\setcounter{section}{-1}


\section{黎曼积分的局限性}


\begin{theorem}
	Let $f_n,f \in C([0,1])$. If $f_n \rightrightarrows f$ on $[0,1]$, then 
	$$\lim_{n\rightarrow \infty}\int_0^1f_n=\int_0^1 f$$
\end{theorem}

Uniform convergence is a sufficient condition for interchanging $lim$ and $\int$, but not necessary.

\begin{example}
	$f_n(x):=x^n,f:=\chi_{\{1\}}$.
\end{example}

Sometimes, the point-wise limit of a sequence of  Riemannian integrable functions is not Riemannian integrable. e.g. the Dirichlet function is the point-wise limit of a sequence of Riemannian integrable functions.







\section{Measuring Sets}

\subsection{集合运算}
\begin{definition}
	\textbf{Symmetric difference} $A\Delta B:=(B\backslash A)\cup (A \backslash B).$
\end{definition}

\begin{proposition}\label{symmetric difference}
Properties of symmetric difference.\par 
\begin{itemize}
	\item $(A\Delta B)\Delta C = A\Delta (B\Delta C)$
	\item $(A\Delta B)\cap C=(A\cap C)\Delta (B\cap C)$
\end{itemize}
\end{proposition}


\begin{definition}
	\textbf{Upper and Lower Limits}
	$$\limsup_{k\rightarrow \infty}A_k:=\bigcap_{k\in \mathbb{N}}\bigcup_{j\geq k}A_j$$
	$$\liminf_{k\rightarrow \infty}A_k:=\bigcup_{k\in \mathbb{N}}\bigcap_{j\geq k}A_j$$
\end{definition}

\begin{proposition}
Properties of upper and lower limits.\par 
	\begin{itemize}
		\item The upper and lower limits exist.
		\item $\limsup A_k=\{\omega:\omega \; lies \; in\; infinite\; many\; A_k.\}$
		\item $\liminf A_k=\{\omega:\omega\; does\; not\; lie\; in\; at\; most\; finite\; A_k \}$
	\end{itemize}
\end{proposition}



\subsection{环、半环和$\sigma$-代数}
我们用集合的交和对称差两种运算在集合族上构造一个环.
\begin{definition}\label{ring}
	称一个集族$\mathfrak{R}$为环,若其对集合交和对称差两种运算封闭.
\end{definition}
\begin{remark}
	根据命题\ref{symmetric difference}, $\mathfrak{R}$确实是一个环.
\end{remark}
\begin{remark}
	由于$A\cup B=(A\Delta B)\Delta(A\cap B)$及$A\setminus B=(A\Delta B)\cap A$,环对于集合的并和减法封闭.
\end{remark}









\begin{definition}
	称一个集族$\mathfrak{S}$为半环,若其包含空集,对交封闭,且仍取$A_1\subset A$,$A_1,A\in \mathfrak{S}$,存在着分解
	\begin{align*}
		A=\bigcup_{k-1}^nA_k \hspace{3ex}A_k\in \mathfrak{S}
	\end{align*}
\end{definition}


半环的有限分解性质可以加强:
\begin{lemma}
	设$\mathfrak{S}$为半环,若$A_1,\dots,A_n,A \in \mathfrak{S}$,且诸$A_k$为$A$两两不交的子集,则可以向$\{A_k\}$中添加$\mathfrak{S}$中的集合$A_{n+1},\dots,A_s$使得
	\begin{align*}
		A=\bigcup_{k=1}^sA_k
	\end{align*}
	\begin{proof}
		使用归纳法:$n=1$时命题显然成立.下考虑$n+1$时情形.
	\end{proof}
\end{lemma}







\begin{definition}
	\textbf{$\sigma$-algebra}. Given a set $X$, $\mathcal{A}\subset P(X)$ is a $\sigma$-algebra if \par 
\begin{itemize}
	\item $\emptyset\in \mathcal{A}$
	\item $A\in \mathcal{A}\rightarrow A^c \in \mathcal{A}$
	\item $\mathcal{A}$ is closed under countable union.
\end{itemize}
\end{definition}



\chapter{Lebesgue Measure}
\section{Measure}
\begin{definition}
	A measure is a set function $\mu: \mathcal{P}(X)\rightarrow [0,\infty]$ satisfying
	\begin{itemize}
		\item $\mu(\phi)=0$
		\item 
	\end{itemize}
\end{definition}



\end{CJK}
\end{document}
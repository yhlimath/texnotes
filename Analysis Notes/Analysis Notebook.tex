\documentclass{article}
\usepackage{amsmath}
\usepackage{mathrsfs}
\usepackage{amssymb}
\usepackage{dcolumn}
\usepackage{graphicx}
\usepackage{fancybox}
\usepackage{amsmath}
\usepackage{amsthm}
\usepackage{amsrefs}
\usepackage{tikz-cd}




\newtheorem{definition}{Definition}
\newtheorem{theorem}{Theorem}
\newtheorem{proposition}{Proposition}
\newtheorem{lemma}{Lemma}
\newtheorem{corollary}{Corollary}
\newtheorem{example}{Example}
\newtheorem{remark}{Remark}

\begin{document}

\title{Analysis Notebook}
\author{heng}
\maketitle

\tableofcontents




\section{Hahn-Banach Theorem}

\begin{theorem}{(Hahn-Banach)}
	Let $E\subset F$ be a subspace. $g$ a functional defined on $E$. Suppose there is a functional $p$ defined on $F$ satisfying 
	
	1) $p(\lambda x)=\lambda p(x)$
	
	2) $p(x+y)\leq p(x)+p(y)$
	
	3) $|g|\leq |p|$ on $E$
	
	then we may extend $g$ to $F$ and the extended functional still satisfies 3).
\end{theorem}

\begin{proof}
	
\end{proof}



\section{Fourier Analysis}
\begin{theorem}[Young's convolution inequality]
	Let $1/p+1/q=1+1/r$ where $p,q,r$ are positive reals. then for $f\in L^p(\mathbb{T})$ and $g\in L^q(\mathbb{T})$, 
	\begin{align*}
		\lVert f*g\rVert_r\leq\lVert f \rVert_p\lVert g \rVert_q
	\end{align*}
	\begin{proof}
	\begin{align*}
		&\lVert f*g \rVert_r^r=\int_{\mathbb{T}}|\int_{\mathbb{T}}f(y)g(x-y)dy |^rdx \\
		\leq&\int_{\mathbb{T}}\int_{\mathbb{T}}\left\{|f(y)|^{p/r}|g(x-y)|^{q/r}|f(y)|^{1-p/r}|g(x-y)|^{1-q/r}dy\right\}^rdx\\
		\leq&\int_{\mathbb{T}}\left\{\lVert f(y)^{p/r}g(x-y)^{q/r} \rVert_r \cdot \lVert f(y)^{1-p/r} \rVert_{\frac{pr}{r-p}}\cdot \lVert g(x-y)^{1-q/r} \rVert_{\frac{qr}{r-q}}\right\}^rdx\quad\mbox{Holder's ineq} \\
		=&\lVert f \rVert_p^{r-p}\cdot \lVert g \rVert_q^{r-q}\int_{\mathbb{T}}\int_{\mathbb{T}}|f(y)|^p|g(x-y)|^qdydx\\
		=&\lVert f \rVert_p^r\cdot \lVert g \rVert_q^r
	\end{align*}
	\end{proof}
	In this, the symmetry of the torus $\mathbb{T}$ is of fundamental importance by ensuring that 
	$$\int_{\mathbb{T}}g(x-y)dy=\int_{\mathbb{T}}g(y)dy$$ and hence
	$$\lVert g(x-y) \rVert_p=\lVert g(y) \rVert_p$$
\end{theorem}










\end{document}
\documentclass{article}
\usepackage{amsmath}
\usepackage{mathrsfs}
\usepackage{amssymb}
\usepackage{dcolumn}
\usepackage{graphicx}
\usepackage{fancybox}
\usepackage{amsmath}
\usepackage{amsthm}
\usepackage{amsrefs}
\usepackage{tikz-cd}

\usepackage{CJK}


\newtheorem{definition}{Definition}
\newtheorem{theorem}{Theorem}
\newtheorem{proposition}{Proposition}
\newtheorem{lemma}{Lemma}
\newtheorem{corollary}{Corollary}
\newtheorem{example}{Example}
\newtheorem{remark}{Remark}

\renewcommand{\today}{\number\year.\number\month.\number\day}






\begin{document}


\title{Workshop 1}
\author{Li Yueheng s2306706}


\maketitle


\section*{Problem 4(c)}

Consider the Fourier series of $f(x)=(x-\frac{1}{2})^2$:
\begin{align*}
	c_n=&\int_{\mathbb{T}}f(x)e^{-i2\pi n}dx \\
	=&\frac{2}{(2\pi n)^2} ,\forall n\neq 0 \\
	c_0=&\frac{1}{12}
\end{align*}
therefore
\begin{align*}
	f(x)\sim \frac{1}{12}+\sum_{n\in\mathbb{Z}}c_ne^{i2\pi nx}=\frac{1}{12}+\sum_{n\geq 1}\frac{con(2\pi nx)}{(\pi n)^2}
\end{align*}
Since right hand side series converges absolutely and each term is a continuous function, it converges to a continuous function. While $f$ is continuous, the function and its Fourier series coincide.

\section*{Problem 4(d)}
take $x=1$ in the equality above to find $\sum_{n\geq 1}\frac{1}{n^2}=\frac{\pi^2}{6}$

\end{document}
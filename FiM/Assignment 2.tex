\documentclass{article}
\usepackage{amsmath}
\usepackage{mathrsfs}
\usepackage{amssymb}
\usepackage{dcolumn}
\usepackage{graphicx}
\usepackage{fancybox}
\usepackage{amsmath}
\usepackage{amsthm}
\usepackage{amsrefs}
\usepackage{tikz-cd}

\usepackage{CJK}


\newtheorem{definition}{Definition}
\newtheorem{theorem}{Theorem}
\newtheorem{proposition}{Proposition}
\newtheorem{lemma}{Lemma}
\newtheorem{corollary}{Corollary}
\newtheorem{example}{Example}
\newtheorem{remark}{Remark}

\renewcommand{\today}{\number\year.\number\month.\number\day}






\begin{document}


\title{}
\author{Li Yueheng s2306706}

\maketitle


\section*{Exercise 1}
\subsection*{a}


\begin{table}[h]
\begin{tabular}{|l|l|l|l|l|}
\hline
Portfolio & $t_0$     & $t_1$          & $t_2$                             & $t_3$                                           \\ \hline
asset         & $50S_0$ & $50S_1$      & $50S_2$                         & $50S_3$                                       \\ \hline
ZCB       & $0$     & $3550$       & $3550e^{r(t_2-t_1)}+3150$       & $3550e^{r(t_3-t_1)}+3150e^{r(t_3-t_2)}$       \\ \hline
total     & $50S_0$ & $50S_1+3550$ & $50S_2+3550e^{r(t_2-t_1)}+3150$ & $50S_3+3550e^{r(t_3-t_1)}+3150e^{r(t_3-t_2)}$ \\ \hline
\end{tabular}
\caption{Portfolio A}
\end{table}

\subsection*{b}
\begin{table}[]
\begin{tabular}{|l|l|l|l|l|}
\hline
Portfolio & $t_0$                                                     & $t_1$ & $t_2$ & $t_3$                                          \\ \hline
forward         & $0$                                                     & /   & /   & $50S_3-K$                                     \\ \hline
ZCB       & $Ke^{r(t_0-t_3)}+3550e^{r(t_0-t_1)}+3150e^{r(t_0-t_2)}$ & /  & /   & $K+3550e^{r(t_3-t_1)}+3150e^{r(t_3-t_2)}$     \\ \hline
total     & $Ke^{r(t_0-t_3)}+3550e^{r(t_0-t_1)}+3150e^{r(t_0-t_2)}$ & /   & /   & $50S_3+3550e^{r(t_3-t_1)}+3150e^{r(t_3-t_2)}$ \\ \hline
\end{tabular}
\caption{Portfolio B}
\end{table}
Assume portfolio B consists of a long forward contract underlying 50 assets $S$ and a ZCB of the amount $T$ that start by $t_0$ and mature at $t_3$.

The value of portfolio B by $t_3$ is 
\begin{align*}
	50S_3-K+Te^{r(t_3-t_0)}
\end{align*}

It is expected that portfolio A and B replicate one another, so we let their values at maturity coincide:
\begin{align*}
	50S_3+3550e^{r(t_3-t_1)}+3150e^{r(t_3-t_2)}=50S_3-K+Te^{r(t_3-t_0)}
\end{align*}
that is,
\begin{align*}
	T=Ke^{r(t_0-t_3)}+3550e^{r(t_0-t_1)}+3150e^{r(t_0-t_2)}
\end{align*}
thus the two portfolios share same value at maturity, and hence they replicate each other.


\subsection*{c}
By the one price law, replicating portfolios have identical current value. A forward contract has $0$ current value, so we have
\begin{align*}
		&T=50S_0 \\
	&K=50S_0e^{r(t_3-t_0)}-3550e^{r(t_3-t_1)}-3150e^{r(t_3-t_2)}
\end{align*}





\section*{Exercise 2}
\subsection*{a}
Suppose current time $t=0$.

Tesco may enter a long forward contract underlying 70 tonnes of flour by maturity time $T= 6$ months, at a forward price 
$$K=70S_0e^{rT}=\pounds 70*1243*e^{0.05*1/2}=\pounds 89250$$
per tonne.

By doing so however the price of flour will rise 6 months later, Tesco will be able to purchase the flour at a fixed price $K$.



\subsection*{b}
Tesco will receive
\begin{align*}
	V_{1/3}=\pounds 70*1294-70*1243*e^{0.05*1/3}=\pounds 2107.7
\end{align*}

\subsection*{c}
Waitrose now pays Tesco $\pounds 1400$ and holds the long forward contract. Here is an arbitrage opportunity:

If Waitrose entered a short forward contract agreeing to sell 70 tonnes of flour after 2 months, this contract would have a forward price
\begin{align*}
	K'=\pounds 70*S_{1/3}e^{0.05*1/6}=\pounds 91338
\end{align*}

By maturity of the two forward contracts, Waitrose would pay $K$ to fulfill the long forward contract and receive $K'$ from the short forward contract. The received flour would be given away immediately. Thus, Waitrose would receive
\begin{align*}
	R=\pounds 1400 +K'-K=\pounds 688
\end{align*}

Whatever the price of flour will be by maturity, this amount of money could surely be earned. Hence the strategy shown above is an arbitrage opportunity for Waitrose.







\end{document}